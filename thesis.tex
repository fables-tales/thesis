% The document class marks this as a thesis, supplying various options that
% control rendering of some standard features (e.g., the cover page).

\documentclass[ % the name of the author
                    author={Sam Phippen},
                % the name of the supervisor (preferably including title)
                supervisor={Dr. Rafal Bogacz},
                % the thesis    title (which cannot be blank)
                     title={Real time voice activity detectors in noisy personal computing environments},
                % the thesis subtitle (which can    be blank)
                  subtitle={},
                % the degree programme (from BSc, MEng, MSci, MSc and PhD)
                    degree={MEng},
                % the year of submission
                      year={2012} ]{thesis}

\usepackage{parskip}
\begin{document}


% =============================================================================

% This section simply introduces the structural guidelines.  It can clearly
% be deleted (or commented out) if you use the file as a template for your
% own thesis: everything following it is in the correct order to use as is.

\section*{Prelude}
\thispagestyle{empty}

A typical thesis will be structured according to a number of standard 
sections described in what follows.  However, it is hard and perhaps
even counter-productive to generalise: the goal of outlining this 
typical structure is {\em not} to be prescriptive, but simply to act 
as a guideline.  In particular, the page counts given are important 
but not absolute: their aim is simply to highlight that a clear, 
concise description is better than a rambling alternative that 
makes it hard to separate important content and facts from trivia.

You can use this document as a \LaTeX-based~\cite{latexbook1,latexbook2}
template for your own thesis by simply deleting extraneous sections (e.g., 
this one); keep in mind that the associated {\tt Makefile} could be of
use, in particular since it also executes \mbox{\BibTeX} to deal with the
bibliography.  If you opt not to do this, which is perfectly acceptible,
a standard cover and declaration of authorship produced online via
\[
\mbox{\url{http://www.cs.bris.ac.uk/Teaching/Resources/cover.html}}
\]

% =============================================================================

% This macro creates the standard UoB title page, with information drawn
% from the document class (meaning it is vital you select the correct
% degree title and so on).

\maketitle

% After the title page (which is a special case in that it is not numbered)
% comes the front matter or preliminaries; this macro signals the start of
% such content, meaning the pages are numbered with Roman numerals.

\frontmatter

% This macro creates the standard UoB thesis declaration; on the hard-copy,
% this must be signed by the author in the space indicated.

\makedecl

% LaTeX will automatically generate a table of contents, and also associated 
% lists of figures, tables and algorithms.  The former is a compulsory part
% of the thesis, but if you do not require the latter they can be suppressed
% by simply commenting out the associated macro.

\tableofcontents
\listoffigures
\listoftables
\listofalgorithms
\lstlistoflistings

% The following sections are part of the front matter, but are not generated
% automatically by LaTeX; the use of \chapter* means they are not numbered.

% -----------------------------------------------------------------------------

\chapter*{Executive Summary}

{\bf A compulsory section, of at most $1$ page} 
\vspace{1cm} 

\noindent
This section should pr\'{e}cis the project context, aims and objectives 
and main contributions and achievements; the same section may be called
an abstract elsewhere.  The goal is to ensure the reader is clear about 
what the topic is, what you have done within this topic, {\em and} what 
your view of the outcome is.

The former aspects should be guided by your specification: essentially 
this section is a (very) short version of what is typically the first 
chapter.  The latter aspects should be presented as a concise, factual 
bullet point list that summarises the main contributions and achievements.  
The points will of course differ for each project, but an example is as 
follows:

\begin{quote}
\noindent
\begin{itemize}
\item I spent $120$ hours collecting material on and learning about the 
      Java garbage-collection sub-system. 
\item I wrote a total of $5000$ lines of source code, comprising a Linux 
      device driver for a robot (in C) and a GUI (in Java) that is 
      used to control it.
\item I designed a new algorithm for computing the non-linear mapping 
      from A-space to B-space using a genetic algorithm, see page $17$.
\item I implemented a version of the algorithm proposed by Jones and 
      Smith in [6], see page $12$, corrected a mistake in it, and 
      compared the results with several alternatives.
\end{itemize}
\end{quote}

% -----------------------------------------------------------------------------

\chapter*{Supporting Technologies}

{\bf A compulsory section, of at most $1$ page}
\vspace{1cm} 

\noindent
This section should present a detailed summary, in bullet point form, of 
any third-party resources (e.g., hardware and software components) used 
during the project.  Use of such resources is perfectly acceptable: the 
goal of this section is simply to be clear about where and how they are 
used.  The content can focus on the project topic itself (rather than, 
for example, including ``I used \mbox{\LaTeX} to prepare my thesis''); 
an example is as follows:

\begin{quote}
\noindent
\begin{itemize}
\item I used the Java {\tt BigInteger} class to support my implementation 
      of RSA.
\item I used a parts of the OpenCV computer vision library to capture 
      images from a camera, and for various standard operations (e.g., 
      threshold, edge detection).
\item I used an FPGA device supplied by the Department, and altered it 
      to support an open-source UART core obtained from 
      \url{http://opencores.org/}.
\item The web-interface component of my system was implemented by 
      extending the open-source WordPress software available from
      \url{http://wordpress.org/}.
\end{itemize}
\end{quote}

% -----------------------------------------------------------------------------

\chapter*{Notation and Acronyms}

{\bf An optional section, of roughly $1$ or $2$ pages}
\vspace{1cm} 

\noindent
Any well written document will introduce notation and acronyms before their 
use, {\em even if} they are standard in some way: this ensures any reader 
can understand the resulting self-contained content.  

Said introduction can exist within the thesis itself, wherever that is
appropriate.  For an acronym, this is typically achieved at the first point 
of use via ``Advanced Encryption Standard (AES)'' or similar, noting the 
capitalisation of relevant letters.  However, it can be useful to include 
an additional, dedicated list at the start of the thesis; the advantage of 
doing so is that you cannot mistakenly use an acronym, for example, 
before defining it.  An example is as follows:

\begin{quote}
\noindent
\begin{tabular}{lcl}
VAD                 &:     & Voice Activity Detection                 \\
VOIP                &:     & Voice Over IP\\
%DES                 &:     & Data Encryption Standard                \\
%                    &\vdots&                                         \\
%${\mathcal H}( x )$ &:     & the Hamming weight of $x$               \\
%${\mathbb  F}_q$    &:     & a finite field with $q$ elements        \\
%$x_i$               &:     & the $i$-th bit of some bit-sequence $x$ \\
\end{tabular}
\end{quote}

% -----------------------------------------------------------------------------

\chapter*{Acknowledgements}

{\bf An optional section, of at most $1$ page}
\vspace{1cm} 

\noindent This thesis would not have been possible without the continual
support of my supervisor Dr Rafal Bogacz.

% =============================================================================

% After the front matter comes a number of chapters; under each chapter,
% sections, subsections and even subsubsections are permissible.  The
% pages in this part are numbered with Arabic numerals.  Note that:
%
% - A reference point can be marked using \label{XXX}, and then later
%   referred to via \ref{XXX}; for example Chapter\ref{chap:context}.
% - The chapters are presented here in one file; this can become hard
%   to manage.  An alternative is to save the content in seprate files
%   the use \input{XXX} to import it, which acts like the #include
%   directive in C.

\mainmatter

% -----------------------------------------------------------------------------

\chapter{Contextual Background}
\label{chap:context}

\vspace{1cm}

The term voice activity detection (sometimes referred to as voice endpoint
location\cite{Tuske}) refers to the algorithms used to distinguish frames of
audio that contain speech from those that contain background
noise\cite{ramirez}. The motivation for this project is that we found, based on
personal experience, that existing VAD algorithms struggle to efficiently
distinguish the noise commonly associated with personal computers (typing and
mouse clicks) from speech.

\section{Applications}

The two main applications of VAD systems are VOIP and Speech Recognition. These
systems both deal with human voice coming in from microphone input, and are
only concerned with what the people that are actually saying, rather than the
entirety of any particular audio sequence. Given these constraints it seems
natural to place a system which only passes through the parts of the audio that
contain human speech.

\subsection{Speech Recognition}

Speech recognition systems are responsible for working out what is being said
in a given sequence of audio. They must be robust to different speakers,
environments, hesitation, stuttering and other human factors. It is often noted
that these systems degrade rapidly in the presence of noise\cite{Moreno} given
that increasingly with systems like Siri\texttrademark\cite{siri} (Apple's
speech recognition system for iOS\texttrademark) we find that people are using
systems that require accurate detection of what is being said in potentially
noisy outdoor environments.

Speech recognition systems can have their accuracy improved\cite{shin} by
employing VAD systems, which are used to locate speech endpoints. The speech
endpoints are then passed into the speech recognition system which then has to
only search the places that the VAD system has flagged up for words, rather
than the entire sequence of audio. If the VAD system is sufficiently sensitive
it may also be able to accurately locate the pauses between words the user
makes, allowing the speech recognition system to assume 1 or very few words
will lie within each part of detected speech.

There is a well known speech recognition benchmark suite called
Aurora\cite{aurora}, over this benchmark suite it was found\cite{ramirez-2}
that by improving the VAD used from the G.729 ITU-T\cite{itut} standard to a
specialised detector that word accuracy (the number of words correctly
recognized) jumped approximately 17\% in the best case.

In the context of this project we believe that our detector may be able to
improve speech recognition through improved voice activity detection, but it is
not our primary goal, as we forsee fewer circumstances where a user is going to
require speech recognition whilst interacting with their computer, than we see
cases where VOIP calls are occurring at the same time as someone is interacting
with the keyboard and mouse inputs.

\clearpage
\subsection{VOIP and Telephony}

When used in a VOIP context VAD systems are typically designed such that they
signal that frames detected as noise are not to be transmitted, preventing the
person at the other end of the call from hearing the noise. Often this is a
component in a system that also includes some noise filtering and compression
the mumble\cite{mumble} system for example includes both an amplitude and
signal-noise ratio based VAD, and then passes the output into the
CELT\cite{celt} codec, here the VAD system silences any noise detected frames
and CELT efficiently compresses the silence to ensure a bandwidth saving.

VAD algorithms allow for large bandwidth savings during both analogue
telephone calls and VOIP calls. The reason for this bandwidth saving is that in
both dialogue (two people speaking) and monologue (one person speaking) much of
the time of the call is occupied by silence. Specifically: 60\% of the time is
occupied by neither person speaking\cite{shah} in a dialogue and 20\% of the
time is occupied by neither person speaking in a monologue.

In the case of a traditional analogue telephone call, the bandwidth saving is
achieved by "Time Assigned Speech Interpolation" whereby a link can carry more
calls by assigning resources only to those calls that currently have someone
speaking on them, causing a significant saving when, for example, speech is
only travelling in one direction (1 slot instead of 2) or allocating no slots
to the call when neither party is speaking\cite{5016247}.

With a VOIP call, a significant bandwidth saving is achieved by nature of the
fact that if transmission were continuous, both parties would be transmitting
at least 64,000 bits per second\cite{ciscovad}. If we assume that for most of
the call only one person is speaking, and that our VAD is sufficiently good at
distinguishing noise from human speech then we will find at least a 50\%
bandwidth saving. Additionally if there are pauses in speech, the VAD will
provide an even greater bandwidth saving.


\section{Problem Definition}

Seeing that both user experience can be improved, and that bandwidth savings
can be achieved through the use of voice activity detection systems, the
question is what can be done to improve these. The specific goal of this
project is to build a voice activity detection system which is robust in the
typical environment of a modern computer user. Whilst these environments do
have a low background noise, they often contain very high amplitude noise
impulses that are caused by the user typing on the keyboard, or interacting
with the mouse.

As is discussed in the implementation section of this project we show that
these systems do not have a high detection accuracy when presented with these
noise impulses. Our aim is to build a VAD system that does provide a high level
of accuracy against this specific type of noise.

We believe that the problem in modern systems exist because most VAD algorithms
are designed for noisy background environments\cite{shin}, but they are not
designed for short, loud noise impulses. These noise impulses are common when a
user is typing on their keyboard. There are two modern consumer VOIP systems
that we used and were found to detect keyboard noise or mouse noise as someone
speaking. Those were: Mumble\cite{mumble}, a low latency system designed for
gamers to communicate whilst playing with each other and
Skype\texttrademark\cite{skype} a well known consumer VOIP solution designed
for single or many person VOIP calls. Especially in the case of mumble it is
essential that keyboard noise is filtered out, as gamers tend to be interacting
with their keyboard continuously whilst playing videogames, possibly drowning
out much of the real communication, or causing annoyance.

The high-level objective of this project is to build a robust voice activity
detector system that can accurately distinguish voice from keyboard and mouse
noise with suitable performance for use in real time systems. More specifically
the concrete aims are:

\begin{enumerate}
    \item Survey existing VAD systems for accuracy against our dataset
    \item Work to develop own algorithm based on machine learning techniques,
          using common features from literature
    \item Build a hangover system which smooths transition from voice to 
          non-voice classes
\end{enumerate}



% -----------------------------------------------------------------------------

\chapter{Technical Background}
\label{chap:technical}

{\bf A compulsory chapter, of roughly $10$ to $20$ pages} 
\vspace{1cm} 

\noindent

%\section {danpagematter}
%This chapter is intended to describe the technical basis on which execution
%of the project depends.  The goal is to provide a detailed explanation of
%the specific problem at hand, and any previous or related work in the area 
%(e.g., descriptions of supporting technologies, existing algorithms that 
%you use, alternative solutions proposed).  
%
%Put another way, after reading this chapter a non-expert reader should have 
%obtained enough background to understand what {\em you} have done, and then
%assess how novel, challenging and rigorous your work is.  You might view an 
%additional goal as giving the reader confidence that you are able to absorb 
%and understand research-level material.

\section {sammatter}

The basic problem any voice activity detector is trying to solve is the
question of whether or not a particular frame of audio does or does not contain
voice. In nearly all recording environments there is continuous background
noise and, when there is speech, the microphone picks up speaking on top of
that. Thusly the output of a VAD system is a single bit (1 or 0 for voice or
noise). In a VAD system we take input from the microphone and then attempt to
determine which of two input classes we are in, corresponding to whether we
should or should not transmit the audio further. We formulate this as a
decision as follows:

\[ decision = \left\{ \begin{array}{ll}
            1 & \mbox{if input like $n + v$: speech present}\\
    0 & \mbox{if input like $n$: no speech present}.\end{array} \right. \] 

Where $n$ represents the presence of background noise and $v$ represents the
presence of conversational speech. This is referred to as additive
noise\cite{sohn} and generally represents a very realistic model of most
environments in which human speech must be differentiated from background noise.

In this project our problem space is slightly different, in that not only do we
have to deal with background noise but we also have to deal with the fact that
keyboard noise is picked up by microphones. This noise is in no way constant:
whilst most people type with a rhythm there are distinct impulses and times
when the amplitude from this noise is very low. In addition to this we may have
to deal with the case where someone is typing and talking at the same time. This
means that our decision model may look something like:

\[ decision = \left\{ \begin{array}{ll}
            1 & \mbox{if input like $n + k + v$: speech present}\\
            0 & \mbox{if input like $n + k$: no speech present}\\
    \end{array} \right. \]

Where $k$ represents the keyboard noise component of the input. This problem is
different as mentioned above, due to the highly non-static nature of keyboard
noise. It is worth noting that this model is still entirely additive, this
model is reasonable because the sources of noise should not destructively
interfere with each other. It is, however, worth noting that the phenomenon of
clipping, the decibel level of the sound sources the microphone can pick up
adding to go above it's input limit, can occur and in this case the additive
models may break down in section \ref{chap:evaluation} we investigate
how important these effects are.



\section{Existing Voice Activity Detection Techniques}


Existing voice activity detection techniques fall into one of two categories.
Either the algorithms contain a number of parameters, and each parameter is
given a threshold (the noise/voice threshold). A decision is made based on the
thresholds over all the parameters as to whether transmission should or should
not occur. In these systems the thresholds can either be static or
adaptive\cite{gokhun}, newer systems tend to deal with the idea that background
noise may be non-stationary and will attempt to adapt their decision based on
long term averages of frames that have been given either noise or voice
classes.

The other major class of voice activity detection algorithms extract parameters
from short "windows"\cite{shin} of audio and then apply a machine learning
algorithm


\subsection{Commonly used features}

One of the most commonly used features in voice activity detection is

% -----------------------------------------------------------------------------

\chapter{Project Execution}
\label{chap:execution}

{\bf A topic-specific chapter, of roughly $20$ pages} 
\vspace{1cm} 

\noindent
This chapter is intended to describe what you did: the goal is to explain
the main activity or activities, of any type, which constituted your work 
during the project.  The content is highly topic-specific, but for many 
projects it will make sense to split the chapter into two sections: one 
will discuss the design of something (e.g., some hardware or an algorithm), 
inc. any rationale or decisions made, and the other will discuss how this 
design was realised via some form of implementation.  

This is, of course, far from ideal for {\em many} project topics.  Some
situations which clearly require a different approach include:

\begin{itemize}
\item In a project where asymptotic analysis of some algorithm is the goal,
      there is no real ``design and implementation'' in a traditional sense
      even though the activity of analysis is clearly within the remit of
      this chapter.
\item In a project where analysis of some results is as major, or a more
      major goal than the implementation that produced them, it might be
      sensible to merge this chapter with the next one: the main activity 
      is such that discussion of the results cannot be viewed separately.
\end{itemize}

\noindent
Note that evidence of ``best practice'' project management (e.g., use of 
version control, choice of programming language and  so on) should only 
be included if there is a clear reason to do so.

\section{Example Section}

This is an example section; 
the following content is auto-generated dummy text.
\lipsum

\subsection{Example Sub-section}

\begin{figure}[t]
\centering
foo
\caption{This is an example figure.}
\label{fig}
\end{figure}

\begin{table}[t]
\centering
\begin{tabular}{|cc|c|}
\hline
foo      & bar      & baz      \\
\hline
$0     $ & $0     $ & $0     $ \\
$1     $ & $1     $ & $1     $ \\
$\vdots$ & $\vdots$ & $\vdots$ \\
$9     $ & $9     $ & $9     $ \\
\hline
\end{tabular}
\caption{This is an example table.}
\label{tab}
\end{table}

\begin{algorithm}[t]
\For{$i=0$ {\bf upto} $n$}{
  $t_i \leftarrow 0$\;
}
\caption{This is an example algorithm.}
\label{alg}
\end{algorithm}

\begin{lstlisting}[float={t},caption={This is an example listing.},label={lst},language=C]
for( i = 0; i < n; i++ ) {
  t[ i ] = 0;
}
\end{lstlisting}

This is an example sub-section;
the following content is auto-generated dummy text.
Notice the examples in Figure~\ref{fig}, Table~\ref{tab}, Algorithm~\ref{alg}
and Listing~\ref{lst}.
\lipsum

\subsubsection{Example Sub-sub-section}

This is an example sub-sub-section;
the following content is auto-generated dummy text.
\lipsum

\paragraph{Example paragraph.}

This is an example paragraph; note the trailing full-stop in the title,
which is common style intended to ensure it does not run into the text.

% -----------------------------------------------------------------------------

\chapter{Critical Evaluation}
\label{chap:evaluation}

{\bf A topic-specific chapter, of roughly $10$ pages} 
\vspace{1cm} 

\noindent
This chapter is intended to evaluate what you did.  The content is highly 
topic-specific, but for many projects will have flavours of the following:

\begin{enumerate}
\item functional testing, inc. analysis of failure cases,
\item performance results, and analysis of said results that draw some 
      form of conclusion,
      and
\item evaluation of options and decisions within the project, and/or a
      comparison with alternatives.
\end{enumerate}

\noindent
This chapter often acts to differentiate project quality: even if the work
completed is of a high technical quality, critical yet objective evaluation 
and comparison of the outcomes is crucial.  In essence, the reader wants to
learn something, so the worst examples amount to simple statements of fact 
(e.g., ``graph X shows the result is Y''); the best examples are analytical 
and exploratory (e.g., ``graph X shows the result is Y, which means Z; this 
contradicts [1], which may be because I use a different assumption'').  As 
such, both positive {\em and} negative outcomes are valid {\em if} presented 
in a suitable manner.

% -----------------------------------------------------------------------------

\chapter{Conclusion}
\label{chap:conclusion}

{\bf A compulsory chapter, of roughly $2$ pages} 
\vspace{1cm} 

\noindent
The concluding chapter of a thesis is often underutilised, in part because
it is often left until close to the deadline and hence does not get enough 
attention.  Ideally, the chapter will consist of three parts:

\begin{enumerate}
\item (Re)summarise the main contributions and achievements, in essence
      summing up the content.
\item Clearly state the current project status (e.g., ``X is working, Y 
      is not'') and evaluate what has been achieved with respect to the 
      initial aims and objectives (e.g., ``I completed aim X outlined 
      previously, the evidence for this is within Chapter Y'').  There 
      is no problem including aims which were not completed, but it is 
      important to evaluate and/or justify why this is the case.
\item Outline any open problems or future plans.  Rather than treat this
      only as an exercise in what you {\em could} have done given more 
      time, try to focus on any unexplored options or interesting outcomes
      (e.g., ``my experiment for X gave counter-intuitive results, this 
      could be because Y and would form an interesting area for further 
      study'').
\end{enumerate}

% =============================================================================

% Finally, after the main matter, the back matter is specified.  This is
% typically populated with just the bibliography.  LaTeX deals with these
% in one of two ways, namely
%
% - inline, which roughly means the author specifies entries using the 
%   \bibitem macro and typesets them manually, or
% - using BiBTeX, which means entries are contained in a separate file
%   (which is essentially a databased) then inported; this is the 
%   approach used below, with the databased being thesis.bib.
%
% Either way, the each entry has a key (or identifier) which can be used
% in the main matter to cite it, e.g., \cite{X}, \cite[Chapter 2}{Y}.

\backmatter

\bibliography{thesis}

% -----------------------------------------------------------------------------

% The thesis concludes with a set of (optional) appendicies; these are the
% same as chapters in a sense, but once signaled as being appendicies via
% the associated macro, LaTeX manages them appropriatly.

\appendix

\chapter{An Example Appendix}
\label{appx:example}

Content which is not central to, but may enhance the thesis can be
included in one or more appendices; examples include, but are not 
limited to

\begin{itemize}
\item lengthy mathematical proofs, numerical or graphical results
      which are summarised in the main body,
\item sample or example calculations, 
      and
\item results of user studies or questionnaires.
\end{itemize}

\noindent
Note that in line with most research conferences, the marking panel 
is not obliged to read such appendices.

% =============================================================================

\end{document}
